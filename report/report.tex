%% 
%% Created in 2018 by Martin Slapak
%%
%% Based on file for NRP report LaTeX class by Vit Zyka (2008)

\documentclass[hidelinks, english]{mvi-report}

\usepackage[utf8]{inputenc}
\usepackage{url}
\usepackage{mathtools}

\usepackage{graphicx}
\usepackage{blindtext}
\usepackage{dirtree}
\usepackage[export]{adjustbox}
\usepackage{array}
\usepackage{bigstrut}
\usepackage{booktabs}
\usepackage{float}
\usepackage{subfigure}
\usepackage[all]{hypcap}
\usepackage[bottom]{footmisc}
\usepackage{caption}
\usepackage{listings}
\usepackage{cleveref}



\graphicspath{{img/}}

\title{Automated machine learning}

\author{Yevhen Kuzmovych}
\affiliation{ČVUT - FIT}
\email{kuzmoyev@fit.cvut.cz}


\newcommand{\subimage}[3][1]{
\subfigure{
    \includegraphics[valign=c, width=#1\textwidth]{#2.#3}
}
}

\newcommand{\smplimage}[3][1]{
\centerline{
    \includegraphics[width=#1\textwidth]{#2.#3}
}
}

\newcommand{\image}[4][1]{
\begin{figure}[H]
    \smplimage[#1]{#2}{#3}
    \caption{#4}
    \label{fig:#2}
\end{figure}
}




\begin{document}

\maketitle

%%%%%%%%%%%%%%%%%%%%%%%%%%%%%%%%%%%%%%%%%%%%%%%%%%%%%%%%%%%%%%%%%%%%%%%%%%%%%%%%
\section{Introduction}

One of the inalienable parts of the data analyst work is selection of the appropriate predictive algorithm for
the given task. In most cases, this part comes down to testing set of selected algorithms on the given dataset or its
subset and selecting the one with the best performance. This process can be automated and improved with prediction of
algorithm quality.

Assuming that each dataset has some hidden properties that could indicate tendency of some algorithms to perform
better than the others, it should be possible to extract those properties and predict algoritms' quality based on them.

There were many attempts that tried to select appropriate meta-features
\cite{sampling-based-relative-landmarks}\cite{statlog}\cite{meta-learning-for-algorithm-selection}. In the framework of
this project, simply obtainable meta-featres used in the StatLog project\cite{statlog} will be combined with landmarks
and relative landmarks described in \textit{Sampling-Based Relative Landmarks: Systematically Test-Driving Algorithms
Before Choosing}\cite{sampling-based-relative-landmarks} to predict algorithms quality.


%%%%%%%%%%%%%%%%%%%%%%%%%%%%%%%%%%%%%%%%%%%%%%%%%%%%%%%%%%%%%%%%%%%%%%%%%%%%%%%%
\section{Methods}



\subsection{Preprocessing}
For this project, only required preprocessing techniques were used:

\begin{itemize}
    \item \textbf{Filling missing data} (because most of the tested models require complete data). NaNs are filled with
    the means in numerical coulmns and most frequent values in categorical.
    \item \textbf{Encoding categorical data}. Nominal data is encoded using \textit{one hot encoding}. Ordinal features
    can be specified with the needed order, labels are then encoded with natural numbers.
\end{itemize}

Implementation is parameterized and can be easely extended with the other techniques.


%%%%%%%%%%%%%%%%%%%%%%%%%%%%%%%%%%%%%%%%%%%%%%%%%%%%%%%%%%%%%%%%%%%%%%%%%%%%%%%%
\section{Outputs}

%%%%%%%%%%%%%%%%%%%%%%%%%%%%%%%%%%%%%%%%%%%%%%%%%%%%%%%%%%%%%%%%%%%%%%%%%%%%%%%%
\section{Conclusion}

%%%%%%%%%%%%%%%%%%%%%%%%%%%%%%%%%%%%%%%%%%%%%%%%%%%%%%%%%%%%%%%%%%%%%%%%%%%%%%%%
\bibliography{reference}


\end{document}
